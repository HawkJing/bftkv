\section{Background}
This section describes some existing technologies used in the system.

\subsection{Quorum Systems}
A variety of quorum systems have been used to manage replicated data /
storage in distributed systems. We briefly describe the original
quorum systems and its extension called Byzantine quorum
systems. Later, we construct a byzantine quorum system based on Web of
Trust.
The system defines two operations, {\em read} and {\em write}, between
a client and a set of servers called a quorum. A quorum system ($QS
\subseteq 2^U$) is a subset of the powerset of all servers ($U$), and
it satisfies a property:
\[
  \forall Q_1, Q_2 \in QS, Q_1 \cap Q_2 \neq \emptyset
\]
\hfill (intersection property)\\

With this property, it is guaranteed that the client always retrieves
the latest value from at least one server.

Melhi and Reiter \cite{Delhi:1} extend the property to handle
Byzantine failure:
\begin{align*}
  \forall Q_1, Q_2 \in QS, \forall B \in BF, Q_1 \cap Q_2 \nsubseteq B
\end{align*}
where $BF \subseteq 2^U$ and $\cup_{B \in BF} B$ is all Byzantine
failure nodes.
Especially when $|Q_1 \cap Q_2| \geq 2b+1$, $QS$ is called a $b$-masking
quorum system. When the client can be dishonet, signed messages are no
longer trustworthy therefore we rely on the quorum system to avoid
equivocation (aka double spending in blockchain terms).

\subsection{Web of Trust}
A web of trust is a directed graph $G = (V, E)$, where $V$ is a set of
nodes (each of which is a pair of unique ID and public key) and $E$ is a
set of trust relationship: when $(v_1, v_2) \in E$, $v1$ {\em trusts}
$v_2$,
i.e., the certificate of $v2$ includes a signature over its public key
with the private key of $v_1$. WoT was introduced by PGP to
authenticate certificates of peers without relying on central
authorities. We use the same mechanism to authenticate not only
end-users' certificates but quorum members' as well.

\subsection{Threshold Cryptosystems}
Threshold cryptosystems play important roles in this system. We use it
not just for fault-tolerance but for improving security.

Shamir's Shared Secret (SSS) is a major tool to construct threshold
cryptographic schemes. The system uses SSS for both password
authentication and DSA / ECDSA signature schemes. Especially for the
latter, the system uses a threshold siganture scheme introduced by
Gennaro et al.\ \cite{Gennaro}. For the threshold password
authentication, the system adopts SRP \cite{srp} with SSS. See
~\ref{threshold} for the details.
SSS uses a $t-1$ degree random polynomial on $\mathbb{Z}_q$
\[
  f(x) = \sum_{i=0}^{t-1}a_ix^i \bmod q
\]
Each {\em share} is $(i, f(i))_{i = 1..n}$. To reconstruct the shared
secret $f(0)$, calculate lagrange interpolate from $t$ responses out
of $n$
\[
    \lambda_j = \prod_{l \in \mathcal{T} \setminus \{j\}}
    i_l / (i_l - i_j) \bmod q
\]
then we get
\[
  f(0) = \sum_{j \in \mathcal{T}} f(j)\lambda_j
\]
where $\mathcal{T}$ is a subset of $\{1..n\}$ and $|\mathcal{T}| =
t$.

To construct a $(t, n)$ threshold scheme, we follow the quorum
threshold, i.e., $n = |Q|, t = n-b$.

For RSA signatures, it may seem straightforward to apply SSS to the
RSA signing process such as:
\[ S = M^{f(0)} = \prod_{i \in \mathcal{T}} M^{f(i)\lambda_i} \bmod N \]
however, since to calculate $\lambda_i$ we need the multiplicative
inverse on $\varphi(N)$ which must be kept as secret as the private
key, we cannot simply appy SSS to RSA signatures. Shoup \cite{shoup}
solved this problem by getting rid of multiplicative inverse all
together but it needs a special construction of the RSA parameters,
which makes it difficult to apply existing keys to the
method. Therefore we use a simple key hierachy to address this
issue. See \ref{threshold} for the details.

For DSA and ECDSA, SSS has preferable properties due to its linearity:
\begin{align*}
f(0) + g(0) &= \sum_{i \in \mathcal{T}} (f(i) + g(i)) \lambda_i \\
f(0) \cdot g(0) &= \sum_{i \in \mathcal{T'}} (f(i) \cdot g(i)) \lambda_i
\end{align*}
where
\begin{align*}
  f(0) &= \sum_{i \in \mathcal{T}} f(i) \lambda_i \\
  g(0) &= \sum_{i \in \mathcal{T}} g(i) \lambda_i
\end{align*}
$T'$ is a subset of $\{1..n\}$ and $|\mathcal{T'}| = 2t$. These
properties make it possible to calculate the DSA / ECDSA signature
$(r, s)$, such that:
\begin{align*}
  r &= g^k \bmod p \bmod q\\
  s &= k^{-1}(x + mr) \bmod q
\end{align*}
from partial signatures $(r_i, s_i)$, where
\begin{align*}
  r_i &= g^{k_i} \bmod p \bmod q\\
  s_i &= k_i^{-1}(x_i + mr_i) \bmod q
\end{align*}
