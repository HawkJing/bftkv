\section{Security Analysis for $\mathcal{TPA}$}
\label{tpa}
\subsection{Correctness}
Assume each $\{Y_i\}_{\{i \in \mathcal{T}\}}$ is correctly calculated,
$G_S$ will be:
\[
  G_S = \prod_{j \in \mathcal{T}}Y_i^{\lambda_j} = g_{\pi}^{a \sum_{j
      \in \mathcal{T}} f(j) \lambda_j} = g_{\pi}^{aS} \bmod p
\]
and by raising $s_i = h(password, t_i)$ to $G_S$, we get:
\[
  X_i = (G_S^{s_i})^{a'} = (g_{\pi}^{Ss_i})^{aa'} \bmod p
\]
which must be the same as $v_i^{a} \bmod p$ for each $i \in
\mathcal{T}$ iff the correct password ($g_{\pi}$) is given at the
step 1. With each $B_i = v_i^{b_i} \bmod p$, the client and servers
share DH keys $K_i = g_{\pi}^{Ss_iaa'b_i} \bmod p$.

\begin{lemma}
\label{tpa1}
From $\{Y_i\}_{i \in \mathcal{L}}$ where $\mathcal{L} \subset \{1..n\}$ and $|\mathcal{L}| <
t$, correct $G_S$ cannot be obtained.
\end{lemma}

\begin{proof}
$S$ cannot be reconstructed from $|\mathcal{L}|$ servers, from the SSS
  property, thus $g_{\pi}^S$ cannot be obtained.
\end{proof}

\subsection{Protocol Analysis}
We look into the protocols by dividing it into two parts: the first
roundtrip (step 1 and 2) is to recover the shared secret $S$, and
the second roundtrip (step 3 and 4) is for key exchange.

$g_{\pi} = \pi^2 \bmod p$, where $\pi$ is a secure hash value over the
password, is a safe generator in $\mathbb{Z}^*_p$ with high
probability. Raise a random exponent $a$ to mask the generator at the
step 1. 

Since $g_{\pi}^{Ss_i}$ is a generator with high probability,
$(g_{\pi}^{Ss_i})^{ab_i} \bmod p$ can be seen as the traditional DH
key, and cannot be distinguished from a random field element under the
DDH assumption \cite{boneh}.

From the correctness lemma, the client correctly obtains
$g_{\pi}^S$ from $t$ servers. Also since it knows the password it can
calculate $g_{\pi}^{h(ID)}$. With the random DH exponent $a$ it can
calculate $g_{\pi}^Ss_ia \bmod p$. At each server $g_{\pi}^{Ss_i}
\bmod p$ is pre-calculated. To obtain $b_i$ from $B_i =
g_{\pi}^{Ss_ib_i} \bmod p$ the client needs to calculate discrete log
on $B_i$. Unless the client can obtain $b_i$ it cannot calculate $K_i$
from a made up DH term. Thus the client has no choice but to calculate
$B_i^a \bmod p$ to obtain $K_i$.

\subsection{Resistance to MiTM}
Assume we have $|\mathcal{L}| < t$ compromised servers and try to
reconstruct $g_{\pi}^S \bmod p$ from ${y_i}$. From the Lemma
~\ref{tpa1} we need at least one $Y_i$, $i \notin \mathcal{L}$, and
unless the server $i$ is not compromised the only way to get $Y_i$ is
to issue the protocol $X = g_{\pi}^a \bmod p$. Unless the attacker
knows $\pi$ all he can do is to guess the password and to keep asking
the server to calculate $Y_i = g_{\pi'}^a \bmod p$ for $\pi'$. If the
guess is wrong $Y'_i$ will not calculate $g_{\pi}^S$ with the
compromised data $\{f(j)\}_{j \in \mathcal{L}}$.
Unless the attacker can calculate $S$ from $g^S \bmod p$ with a
fabricated $g$, he cannot calculate $g_{\pi'}^S \bmod p$ in
order to brute force for $g_{\pi'} \notin \{g^e | e = 1..p-1\}$.

\subsection{Resistance to offline dictionary attacks}
Under the same condition of MiTM, what the client can obtain is:
\begin{align*}
  y_i &= f(i) \\
  v_i &= g_{\pi}^{Ssi} \bmod p
\end{align*}
for $i \in \mathcal{L}$. We also get $G_S = g_{\pi}^{aS} \bmod p$ from
the step 2, and $Z_i = E_{h(K_i)}(Pi, X_i||Y_i)$ from the step
4. $B_i$ will not contribute anything the attack as we already have
$v_i$.
To brute force on $v_i$ there are two possible ways:
\begin{enumerate}
\item $\exists S \in \mathbb{Z}_q$, check if
  $v_i \stackrel{\text{\tiny ?}}{=} (g_{\pi}^{s_i})^S \bmod p$
  for each $g_{\pi}^{s_i}$ \\
\item $\exists g_{\pi} \in \mathbb{Z}^*_q$, check if
  $v_i \stackrel{\text{\tiny ?}}{=} (g_{\pi}^S)^{s_i} \bmod p$
  for each $s_i$
\end{enumerate}
For the former case we need $S$ and for the latter case we need
$g_{\pi}^S$. From the protocol 1, we can obtain either $g^S \bmod p$
for $\forall g \in \mathbb{Z}$ or $g_{\pi}^{aS} \bmod p$. Obtaining
$S$ is simply the {\sf DLOG} problem. $g_{\pi}^S \bmod p$ can be
obtained iff $g$ happens to be $g_{\pi}$ or a legitimate client
chooses $a = 1$ which must be excluded.
If $g_{\pi}^S$ is obtained, the dictionary attack is possible on
$s_i$, however, it is difficult to obtain $g_{\pi}^S$ from $G_S$ which
is calcualted by a legitimate client as $a$ has been randomly
chosen. Attackers can choose $a = 1$ at the step 1 and they can obtain
$g_{\pi'}^S$ but they need to start over the protocol to get the
correct $g_{\pi}^S$.
