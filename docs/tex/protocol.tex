\section{Protocols}
\label{Protocols}
The system uses Phalanx \cite{Delhi:2} for the underlying {\sf
read/write} operations. To maintain a graph for the quorum system, we
extend the protocols with {\sf join/leave}. Also {\sf register}
protocol is used to get quorum certificates with the threshold
password authentication.

{\em Notations}:
$\langle x, t, v, s_C, S \rangle$ is an ordered set that denotes the protocol
data. $s_C$ and $S$ can be omitted.

\begin{tabular}{cl}
  $x$ & \begin{minipage}[t]{0.8\columnwidth}%
    is the variable and an arbitrary length octet
    string. The variable will be the ``key'' of the key-value store.%
  \end{minipage}\\
  $t$ & \begin{minipage}[t]{0.8\columnwidth}%
    is the timestamp and a 64-bit non-negative
    integer. $2^{64}-1$ is a special value to denote that the variable is
    no longer able to be updated.%
  \end{minipage}\\
  $v$ & \begin{minipage}[t]{0.8\columnwidth}%
    is the value and an arbitrary length octet string.%
  \end{minipage}\\
  $s_C$ & \begin{minipage}[t]{0.8\columnwidth}%
    is an object of the signature and quorum certificate
    of a client $C$. $s_C.sig$ is the signature over $\langle x, t, v
    \rangle$ signed by the private key of $C$. $s_C.cert$ is the quorum
    certificate of $C$. $s_C.cert.ID$ is the unique ID to identify each
    node.%
  \end{minipage}\\
  $S$ & \begin{minipage}[t]{0.8\columnwidth}%
    is an unordered set of the signature object. Each $S_i
    \in S$ has the
    same structure of the $s_C$. The signers must be quorum members.%
  \end{minipage}\\
\end{tabular}

\subsection{Read / Write}
\label{rw}
{\sf read/write} protocols are done between a client ($C$) and a
quorum ($Q$). A quorum is chosen from a quorum system ($QS \subseteq
2^U$) which is a subset of the powerset of all nodes ($U$).  To write
a value $v$ into a variable $x$, we follow the write protocol
specified in Phalanx.\\

[ {\sf write} ]
\begin{align*}
  C :& \text{ choose a quorum from quorum cliques} \\
     & Q \in QC \\
  C \rightarrow Q :& \text{ {\sf get timestamp}}(x) \\
  C \leftarrow Q_i :& \text{ } t_i \\
  C :& \text{ collect } 2b+1 \text{ timestamps: } \\
     & \text{ } \{t_i\}_{i \in \mathcal{T}}, |\mathcal{T}|=2b+1,
       \mathcal{T} \subseteq Q \\
  C :& \text{ choose the maximum timestamp } \\
     & \text{ } t = max(t_i) + 1 \\
  C \rightarrow Q :& \text{ {\sf get signature}}(\langle x, t, v, s_C \rangle),\\
     & \text{ where } s_C = (Sign_C(\langle x, t, v \rangle), Cert_C) \\
  Q_i :& \text{ verify $s_C$ with $C$'s quorum certificates} \\
  Q_i :& \text{ check the TOFU policy} \\
  C \leftarrow Q_i :& \text{ } S_i = \{\langle x, t, v, s_C
  \rangle\}_{Q_i \in Q} \\
  C :& \text{ collect signatures from $|\mathcal{T}|$ members} \\
     & \text{ } S = \{S_i\}_{i \in \mathcal{T}} \\
  C :& \text{ choose a quorum from } Q' = U \setminus QC \\
  C \rightarrow Q' :& \text{ write}(\langle x, t, v, s_C, S \rangle) \\
  Q'_i :& \text{ verify the signature set $S$} \\
  Q'_i :& \text{ do the equivocation check} \\
  Q'_i :& \text{ store } \langle x, t, v, S_C, S \rangle
\end{align*}

[ {\sf read} ]
\setcounter{equation}{0}
\begin{align*}
  C :&\; \text{choose a quorum } Q \in U \setminus QC \\
  C \rightarrow Q :&\; read(x) \\
  C \leftarrow Q_i :&\; M_i = \langle x, t, v, s_C, S \rangle \\
  C :&\; \text{collect } f + 1 \text{ responses} \\
  C :&\; \text{verify the signature set $S$} \\
  C :&\; \text{do the equivocation check} \\
  C :&\; \text{choose the latest timestamp } \\
     &\; t_L = max(M_i.t) \\
  C :&\; Q' = \{Q_i \subset Q | M_i.t < t_L\} \\
  C \rightarrow Q' :&\; \text{{\sf write back}}(\langle x, t_L, v, s_C, S \rangle)
\end{align*}


\subsection{Join / Leave}
Any node can join / leave anytime by sending its quorum certificate to
nodes it trusts. The node received the request verifies the
certificate and adds it to the local graph which will be returned to
the caller node. To leave the network, a node broadcasts its quorum
certificate to the quorum it belongs to.
The node that has received the graph constructs its own local graph
from the graphs, then sends the join request to nodes that have not
been connected.
See algorithm {\sf Join} \ref{Join}.

\subsection{Register}
\label{register}
Each node generates its own public/private key pair, then sends the
public key part to a quorum to get a quorum certificate. Each quorum
member keeps all certificate issued to clients along with a partial
password secret. If it finds a certificate request that has the same
ID as one of the stored certificates it will sign it only if the
password authentication is done. The client will get a ``proof'' of
the authentication when it is finished.
See algorithm {\sf Register} \ref{Register}

\subsection{Authenticate}
\label{authenticate}
{\sf auth} protocol is done in advance of {\sf read / write /
  register} to get a proof and each subsequence protocol verifies it
before processing the request. The follwing diagram is for a quick
review of $\mathcal{TPA}$ \ref{auth}.

\begin{align*}
  C &: \pi \leftarrow h(pw), \;
      g_{\pi} \leftarrow \pi^2, \;
      a \xleftarrow{\mathcal{R}} \mathbb{Z}_q \\
  C \rightarrow Q &: X := g_{\pi}^a \bmod p \\
  C \leftarrow Q_i &: Y_i := X^{y_i} \bmod p, t_i \\
  C &: G_S \leftarrow \textstyle \prod_{j \in \mathcal{T}}Y_i^{\lambda_j} \bmod p, \\
    & \; s_i \leftarrow h(pw, t_i), \;
      a' \xleftarrow{\mathcal{R}} \mathbb{Z}_q \\
  C \rightarrow Q_i &: X_i := G_S^{s_ia'} \bmod p \\
  Q_i &: K_i \leftarrow X_i^{b_i} \bmod p, \;
        b_i \xleftarrow{\mathcal{R}} \mathbb{Z}_q \\
  C \leftarrow Q_i &: B_i := v_i^{b_i} \bmod p,
                     Z_i := E_{K_i}(P_i, \dots) 
\end{align*}

At the end of the protocol, $C$ obtains the proof $P = \{P_i\}_{i \in
  \mathcal{T}}$.
