\section{Security Analysis}
We look into attacks against the fundamental property:
$READ(Q_1,x) = READ(Q_2,x)$ for $\forall Q_1, Q_2 \in QS$, which is
known as equivocation.  The best that attackers can do is divide a
clique into two sets and ask each set to sign $\langle x,t,v \rangle$
and $\langle x,t,v' \rangle$ separately. Then do the {\em write}
protocol for the target nodes with collected signature sets $S$ and
$S'$. Honest servers will refuse the request because it does not
satisfy the basic $b$-masking quorum condition: $|S| \geq b+1$. But
with $b$ colluding nodes, the attack will succeed.

\newcommand{\slice}[4]{
  \pgfmathparse{0.5*#1+0.5*#2}
  \let\midangle\pgfmathresult

  % slice
  \draw[thick,fill=black!10] (0,0) -- (#1:1) arc (#1:#2:1) -- cycle;

  % outer label
  \node[label=\midangle:#4] at (\midangle:1) {};

  % inner label
  \pgfmathparse{min((#2-#1-10)/110*(-0.3),0)}
  \let\temp\pgfmathresult
  \pgfmathparse{max(\temp,-0.5) + 0.8}
  \let\innerpos\pgfmathresult
  \node at (\midangle:\innerpos) {#3};
}

\begin{tikzpicture}[scale=1.5]

\newcounter{a}
\newcounter{b}
\foreach \p/\t/\l in {25//,40/$H_1$/Honest nodes, 20/$F$/Faulty nodes,
40/$H_2$/Honest nodes}
  {
    \setcounter{a}{\value{b}}
    \addtocounter{b}{\p}
    \slice{\thea/100*360}
          {\theb/100*360}
          {\t}{\l}
  }

\end{tikzpicture}

The maximum number of signatures dishonest clients can get is
$b+(n-b)/2$. Therefore, to overcome the attack we need
\[ n-b > b+(n-b)/2 \Rightarrow n > 3b \]

\subsubsection*{Detecting equivocation on read}
Even if the number of faulty nodes exceeds the above threshold, the
system can detect malicious actions with the following probability
\begin{align*}
  F_p &= Pr[Q \cap H_1 \neq \emptyset \wedge Q \cap H_2 \neq
        \emptyset] \\
      & = 1 - Pr[Q \subseteq F \cup H_1] \\
      & = 1 - ((n + f) / 2n)^{|Q|}
\end{align*}
when $f > b$ and $|Q| < (f + n)/2$, where $f$ is the number of the
faulty nodes, assuming the sizes of $H_1$ and $H_2$ are the same.

In the case of $f \le b$ the detection rate is always 100\% because it is
guaranteed that clients can always find a valid value and anything
other than that is the result of malicious actions. When the number of
faulty nodes exceeds the threshold, i.e., $f \geq b$, it will be possible that
the client cannot detect all malicious actions. Since the minimum
quorum size is $(n-1)/3$ it does not need to consider the case where
$|Q| < n/3$. Also, if the size exceeds $(f+n)/2$ any quorum always
includes at least one node from each $H_i$ which makes the detection
rate 100\%. \\

\begin{tikzpicture}[xscale=6,yscale=6]
  \draw [<->] (0,0.9) -- (0,0) -- (1.0,0);
  \node [below right] at (1,0) {$|Q|$};
  \node [left] at (0,{1-pow(17/300,0.7)}) {$1$};
  \node [left] at (0,0) {$0$};
  \node [below] at (0.3,0) {$n/3$};
  \node [below] at (0.7,0) {$(f+n)/2$};
  \draw[dashed, domain=0:0.3] plot (\x, {1-pow(17/300,\x)});
  \draw[thick, domain=0.3:0.7] plot (\x, {1-pow(17/300,\x)});
  \draw[dashed, domain=0.7:1.0] plot (\x, {1-pow(17/300,0.7)});
\end{tikzpicture}

For example, assume we choose a quorum $|Q| = 3b+1$ out of $n = 4b+1$,
which is the default setup of the kv quorum system, the detection rate
is 100\% up to $f = 2b$ failure nodes. \\

\begin{tikzpicture}[xscale=6,yscale=6]
  \draw [<->] (0,1.05) -- (0,0) -- (1.05,0);
  \node [left] at (0,1) {$1$};
  \node [left] at (0,0) {$0$};
  \node [below] at (0.3,0) {$n/3$};
  \node [below] at (0.7,0) {$2n/3$};
  \node [below] at (1,0) {$n$};
  \node [right] at (1.05,0) {$f$};
  \draw[thick, smooth] (0,1) to (0.6,1) to [out=0,in=90] (1,0);
\end{tikzpicture}
