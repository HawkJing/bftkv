\begin{abstract}
  \noindent
  Most distributed key-value stores are tolerant to only benign
  failure, which makes it difficult to run the system in an insecure
  environment (i.e., the Internet) as a single point of failure could
  compromise the whole system.  Such systems not only are vulnerable to
  malicious attacks, but also need to rely on centralized authorities to
  protect data integrity.
  We developed a distributed key-value store that is tolerant
  to Byzantine faults, keeping it in mind to run the system over the
  public Internet without any central
  authorities.  While data integrity is secured among distributed
  nodes, anyone can join / leave the network freely and can see all
  transactions.
  The system provides not only a robust key-value store but also
  authentication, encryption and signing features with a threshold
  cryptosystem. Integrating encryption and signature schemes with
  a Byzantine fault-tolerant protocol is much more robust than using
  separated KMS and PKI with an ordinary distributed key-value store,
  as it maximizes a benefit of distributed systems which fit very well
  with threshold cryptography. There is no single point of failure in
  the system.
  The system by itself is useful as a secure key-value store, but those
  properties such as Byzantine fault-tolerance, transparency,
  distributed global storage and threshold cryptosystem can make the
  system an ideal building block for blockchain technologies.
\end{abstract}
