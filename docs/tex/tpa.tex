\section{Security Analysis for $\mathcal{TPA}$}
\label{tpa}
\subsection{Correctness}
Assume each $\{Y_i\}_{\{i \in \mathcal{T}\}}$ is correctly calculated,
$G_S$ will be:
\[
  G_S = \prod_{j \in \mathcal{T}}Y_i^{\lambda_j} = g_{\pi}^{a \sum_{j
      \in \mathcal{T}} f(j) \lambda_j} = g_{\pi}^{aS} \pmod p
\]
and by raising $s_i = H(pw, u_i)$ to $G_S$, we get:
\[
  X_i = (G_S^{s_i})^{a'} = (g_{\pi}^{Ss_i})^{aa'} \pmod p
\]
which must be the same as $v_i^{aa'} \bmod p$ for each $i \in
\mathcal{T}$ iff the correct password ($g_{\pi}$) is given at the
step 1. With each $B_i = v_i^{b_i} \bmod p$, the client and servers
share DH keys $K_i = g_{\pi}^{Ss_iaa'b_i} \bmod p$.

\begin{lemma}
\label{tpa1}
From $\{Y_i\}_{i \in \mathcal{L}}$ where $\mathcal{L} \subset \{1..n\}$ and $|\mathcal{L}| <
t$, correct $G_S$ cannot be obtained.
\end{lemma}

\begin{proof}
$S$ cannot be reconstructed from $|\mathcal{L}|$ servers, from the SSS
  property, thus $g_{\pi}^{aS}$ cannot be obtained whatever $a$ is.
\end{proof}

\subsection{Protocol Analysis}
We look into the $\mathcal{TPA}$ protocols by dividing it into two
parts: the first roundtrip (step 1 and 2) is to recover the shared
secret $S$, and the second roundtrip (step 3 and 4) is for key
exchange.

$g_{\pi} = \pi^2 \bmod p$, where $\pi$ is a secure hash value over the
password, is a safe generator in $\mathbb{Z}^*_q$ with high
probability. Raise a random exponent $a$ to mask the
generator. Servers blindly raise its own share $y_i$ to whatever $X =
g_{\pi}^a$ is. As long as $p$ is a safe prime $X^{y_i} \bmod p$ does
not reveal $y_i$ whatever $X$ is.

The second roundtrip can be seen as the Elgamal encryption with a
generator $G_S^{s_i}$ where $s_i := H(pw, u_i)$. $u_i$ is a salt
generated uniquely for each server. The encrypted data is sent in the
next roundtrip.  Another random exponent $a'$ is applied to make it
difficult to do offline dictionary attacks over $s_i$.

The last roundtrip (step 4 and 5) is to check if both parties have
shared a random session key and to obtain a proof. The client receives
the encrypted data iff the MAC is verified at the server, otherwise it
will receive $\bot$.

\subsection{Reduction to Elgamal Encryption}
Since $\mathbb{Z}_p^*$ is a cyclic group, there exist $g \in
\mathbb{Z}_p^*$ and $x \in \mathbb{Z}_p$ such that $g_{\pi} = g^x
\pmod p$ therefore in the step 3, $X_i = G_S^{s_ia'}$ can be
seen as a $DH$ term: $g^{x'}$ where $x' = xSs_iaa' \bmod q$. The
session key on the server side is the same as the standard Elgamal
encryption, i.e., $(g^{x'})^{b_i}$.
At the step 4, the server returns $B_i = v_i^{b_i}$ where $v_i =
g^{x'/aa'} \pmod p$. For any $aa' \in \mathbb{Z}_q$ there exist a
multiplicative inverse in $\mathbb{Z}_q$ such that $v_i^{aa'} = g^{x'}
\pmod p$. Therefore we get the standard Elgamal key such as $B_i^{aa'}
= (g^{x'})^{b_i} \pmod p$ on the client side.

\subsection{Resistance to offline dictionary attacks}
Assume we have $|\mathcal{L}| < t$ compromised servers and try to
reconstruct $g_{\pi}^S$ from $\{y_i\}_{i \in \mathcal{L}}$.
From the Lemma ~\ref{tpa1} we need at least one $Y_i$, $i \notin
\mathcal{L}$, and unless the server $i$ is compromised the only
way to get $Y_i$ is to issue the protocol $X = g_{\pi}^a \bmod
p$ in the step 1.
Unless the attacker knows $\pi$ all he can do is to guess the
password $\pi'$ and to keep asking the server to calculate $Y'_i =
g_{\pi'}^a \bmod p$. If the guess is wrong $Y'_i$ will not
calculate $g_{\pi}^S$ correctly with the compromised data $\{f(j)\}_{j
\in \mathcal{L}}$.
Unless the attacker can calculate $S$ from $g^S \bmod p$ with a
fabricated $g$, he cannot calculate $g_{\pi'}^S \bmod p$ in
order to brute force for $g_{\pi'} \in \{g^e | e = 1..p-1\}$.

Next, we look into dictionary attacks for {\em share} that is
revealed from compromised servers, i.e.,
\begin{align*}
  y_i &= f(i) \\
  v_i &= g_{\pi}^{Ss_i} \bmod p \\
\end{align*}
for $i \in \mathcal{L}$. We also get $G_S = g_{\pi}^{aS} \bmod p$ from
the step 2. $B_i$ will not contribute anything to the attack as they already
have $v_i$.
To brute force on $v_i$ there are two possible ways:
\begin{enumerate}
\item $\exists S \in \mathbb{Z}_q$, check if
  $v_i \stackrel{\text{\tiny ?}}{=} (g_{\pi}^{s_i})^S \pmod p$
  for each $g_{\pi}^{s_i}$ \\
\item $\exists g_{\pi} \in \mathbb{Z}^*_q$, check if
  $v_i \stackrel{\text{\tiny ?}}{=} (g_{\pi}^S)^{s_i} \pmod p$
  for each $s_i$
\end{enumerate}
For the former case we need $S$ and for the latter case we need
$g_{\pi}^S$. From the protocol 1, we can obtain either $g^S \bmod p$
for $\forall g \in \mathbb{Z}$ or $g_{\pi}^{aS} \bmod p$. Obtaining
$S$ is simply the {\sf DLog} problem. $g_{\pi}^S \bmod p$ can be
obtained iff $g$ happens to be $g_{\pi}$ or a legitimate client
chooses $a = 1$ which must be excluded.
If $g_{\pi}^S$ is obtained, the dictionary attack is possible on
$s_i$, however, it is difficult to obtain $g_{\pi}^S$ from $G_S$ which
is calcualted by a legitimate client as $a$ has been randomly
chosen. Attackers can choose $a = 1$ at the step 1 and they can obtain
$g_{\pi'}^S$ but they need to start over the protocol to get the
correct $g_{\pi}^S$.
