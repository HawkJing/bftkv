\section{Applications}
We discuss some potential appliations of BFTKV.

\subsection{Decenteralized PKI with DKMS}

User authentication is a long-standing problem for end-to-end
systems. Even if we have semantically secure cryptographic protocols
to exchange data between users, if it was with a wrong one, the whole
security system would not make sense. On the other hand, once we have
a robust user authentication scheme, we can build up many kinds of
security systems on top of that, such as PGP and Signal. Our goal is
to construct an infrastructure to exchange public keys that represent
users' identities. Exchanging public keys can be done in person,
using a QR code, confirming the fingerprint of public keys, etc. Those
methods seem to be relevant for some situations, such as sending
money. Also, public key infrastructures using central authorities,
such as X.509 which is based on chain of trust, are widely used. A PKI
like X.509, however, still have a problem when issuing a certificate
to each end user. CAs issue certificates to corporates, organizations,
and individuals based on trustworthiness of requesters but for end
users whose authenticity is not easy to be proven, we have the same
basic issues. From end users' point of view, blindly relying on a
central authority based on its authenticity is no longer secure and
contradict the end-to-end philosophy.

Our proposed PKI does not ``strongly'' rely on central authorities,
yet it does not require to exchange public keys in person. Here are
the high level system requirements:
\begin{itemize}
\item Scalability -- the system can grow without affecting the current running services
\item Transparency -- anyone can monitor every system activity
\item Quantifiability -- security and efficiency can be formally analyzed
\item Robustness -- the system has to recover from erroneous situations by itself
\item Privacy -- the system should not reveal unnecessary information about users
\item Non-interactivity -- a client may not be able to interact peers
before sending a message. This particular requirement makes it
difficult to design a system that guarantees the ``what I saw is what
you see'' concept. SMTP, for example, is not a mutual explicit
authentication protocol. When an email is encrypted then sent out, if
it is encrypted with a wrong key, it will be too late -- someone in
the middle could read the email when the recipient receives the email
and notice that the encrypted email is not actually for her.
\end{itemize}

\subsection{Consensus Mechanism for Blockchain Technologies}
Key-value stores can be suitable to store transactions. Without
knowing the key it will be difficult to access the value -- in this
sense it can be less transparent than other ledgers such as hash
chain. To keep the transactions we can simply use the hash value of
the whole transaction data. As every transaction should have a public
key (``address'') and transaction ID in the sense of bitcoin
Tx, the hash value should be unique. We write the value with
the WRITE ONCE permission.
