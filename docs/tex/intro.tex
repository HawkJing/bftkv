\section{Introduction}
How to reach a consensus with Byzantine type failure is the main
problem of blockchain technologies. Bitcoin blockchain uses PoW (Proof
of Work) with incentives \cite{bitcoin}. Majority of public blockchain
(or permission-less blockchain) has the same type of consensus
mechanism. Another way to establish a consensus is to use BFT
(Byzantine Fault-Tolerant) protocols, which is a major mechanism for
local/enterprise blockchain (or permissioned system). Some PoS (Proof
of Stake) type blockchain technologies use BFT as well to improve the
transaction rate.

The proposed system is based on the Byzantine quorum system introduced
by Malhi and Reiter \cite{Delhi:1}, and a BFT protocol proposed by the
same authors \cite{Delhi:2}. We construct a $b$-masking quorum system
based on WoT (Web of Trust) graph. We also introduce quorum
certificates combined with a threshold authentication scheme to
protect data from unauthorized mutation. Unlike a centralized PKI such
as X.509, quorum certificates are verified according to a dynamic
graph constructed independently at each node. With the strong
authentication scheme and flexible certificate mechanism, the system
allows anyone to join / leave the network freely without any
authorization process.

Another key aspect of blockchain technologies is the global
distributed ledger. Bitcoin blockchain uses hash chain -- transactions
in a specific period are all hashed together with the hash value of
the previous block. All bitcoin nodes maintain copies of only one
global chain, i.e., global ledger.
The proposed system uses a distributed key-value store with the TOFU
(Trust on First Use) policy, that is, the first use of a key locks out
others from mutating the value associated with the key. Only the
user who has written the key-value first will posses the right to
update the value. Also it supports WRITE ONCE permission which
guarantees that once a value is written it will never be modified in
any way. This special permission will be preferable for applications
like the global ledger which must be tamper-proof.
The system does not guarantee absolute consistency. Also the order of
transactions for different keys does not matter. The system does not
even provide eventual consistency among individual nodes. Instead, it
has a property such that: $READ(Q_1, x) = READ(Q_2, x), \forall Q_1,
Q_2 \in QS$, that is, the consistency is established collectively with
a quorum system ($QS$).

Smart contracts play an important role in some blockchain technologies
such as Ethereum and Hyperledger Fabric. As the proposed system is a simple
key-value store, it does not provide a platform for smart contracts at
the moment. Also, the system itself does not provide any incentive
mechanism to discourage nodes to do malicious actions, or to encourage
to participate the consensus process. The system has a robust
revocation scheme but it does not answer for a question like why would
one want to run a node? Although all transactions keep a proof of good
or bad actions for each node therefore it will be straightforward to
map it to economic incenstives and penalties, we defer fintech
discussions to an application layer.
